\section{Численный метод решения задачи}

\textbf{Вариант 2:} набор данных 1, равномерная сетка, максимум-норма.

Для решения задачи был использован метод \textit{скорейшего спуска} (на первой итерации) и метод \textit{сопряженных градиентов} (на последующих итерациях) на равномерной сетке при заданном количестве точек $N_x$ и $N_y$ по осям $Ox$ и $Oy$ соответственно:
\begin{gather}x_i = 3\frac{i}{N_x}, y_j = 3\frac{j}{N_y}.\end{gather}

Полностью численный метод решения задачи описан в \cite{task_pdf}.

Опишем условие остановки итерационного процесса. Пусть $P^{(n)}= [p_{ij}^{(n)}]$~-- приближенное решение, полученное на итерации $n$. Максимум-норма:
\begin{gather}\parallel p\parallel = max_{\substack{0 < i < N_1 \\ 0 < j < N_2}} |p_{ij}|,\end{gather}
где $P = [p_{ij}]$.

Итерационный процесс останавливается, как только:
\begin{equation}
  \begin{gathered}\parallel P^{(n)} - P^{(n - 1)}\parallel < \epsilon,\end{gathered}
  \label{eq:stop}
\end{equation}
где $\epsilon = 0.0001$.

\clearpage
